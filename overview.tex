\chapter{A brief Overview of Proof Systems}
  After defining the important notions for this thesis, we will give a brief overview of the most important results in the field of optimal proof systems.

  One basic lemma that is widely used establishes a connection between optimal proof systems and polynomial many-one-reducibility. Later in this thesis, we will use it to proof corollary \ref{corHardSets}. The following proof is mainly taken from Köbler et al. \cite{KMT03}.

  \begin{lemma}
    If \(A\) has a (p-)optimal proof system and if \(B \redmany A\), then \(B\) has a (p-)optimal proof system, too.
  \end{lemma}

  \begin{proof}
    Let \(h\) be a p-optimal proof system for \(A\) and let \(B\) many-one reduce to \(A\) via \(f \in \FP\), that is \(x \in B \Leftrightarrow f(x) \in A\). Then \(h'\) defined by
      \[ 
        h'(\langle x, w \rangle) =
        \begin{cases}
          x & (h(w) = f(x)), \\
          \perp & (\text{otherwise}), \\
        \end{cases}
      \]
    is a proof system for \(B\), as \(h(w) = f(x) \in A\) is equivalent to \(x \in B\). To show \(h'\) is optimal, let \(g'\) be a proof system for \(B\). In order to obtain a proof system for \(A\), let \(g\) be
      \[
        g(bw) =
        \begin{cases}
          h(w) & (b = 0), \\
          f(g'(w)) & (b = 1). \\
        \end{cases}
      \]
    Since both \(h(w)\) and \(f(g'(w))\) are in \(A\) and \(h\) is a proof system for \(A\), \(g\) is also a proof system for \(A\). As \(h\) is p-optimal, there is a function \(t \in \FP\) translating \(g\)-proofs to \(h\)-proofs implying that
    \[
      h(t(1w)) = g(1w) = f(g'(w)).
    \]
    This implies \(h'(\langle g'(w), t(1w) \rangle ) = g'(w)\). Hence, \(h'\) p-simulates \(g'\).
  \end{proof}

  A main motivation to study proof systems is given by the following theorem \cite{KMT03}.

  \begin{theorem}
   \(\NP = \coNP\) if and only if a polynomial bounded proof system for \(\TAUT\) exists.
  \end{theorem}

  According to Köbler et al., one tried to separate \(\NP\) from \(\coNP\) by studying more and more powerful proof systems, showing that they are not polynomially bounded \cite{KMT03}.




  