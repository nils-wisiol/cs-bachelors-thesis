\chapter{A brief Overview of Proof Systems} \label{chpOverview}
  After defining the important notions for this thesis, we will give a brief overview of some important results in the field of optimal proof systems.

  One basic lemma that is widely used formalizes a part of the connection between optimal proof systems and polynomial many-one-reducibility. Later in this thesis, we will use it to proof corollary \ref{corHardSets}. The following proof is mainly taken from Köbler et al. \cite{KMT03}.

  \begin{lemma} \label{lemManyOneProofSystem}
    If \(A\) has a (p-)optimal proof system and if \(B \redmany A\), then \(B\) has a (p-)optimal proof system, too.
  \end{lemma}

  \begin{proof}
    Let \(h\) be a p-optimal proof system for \(A\) and let \(B\) many-one reduce to \(A\) via \(f \in \FP\), that is \(x \in B \Leftrightarrow f(x) \in A\). Then \(h'\) defined by
      \[
        h'(\langle x, w \rangle) =
        \begin{cases}
          x & (h(w) = f(x)), \\
          \perp & (\text{otherwise}), \\
        \end{cases}
      \]
    is a proof system for \(B\), as \(h(w) = f(x) \in A\) is equivalent to \(x \in B\). To show \(h'\) is optimal, let \(g'\) be a proof system for \(B\). In order to obtain a proof system for \(A\), let \(g\) be
      \[
        g(bw) =
        \begin{cases}
          h(w) & (b = 0), \\
          f(g'(w)) & (b = 1). \\
        \end{cases}
      \]
    Since both \(h(w)\) and \(f(g'(w))\) are in \(A\) and \(h\) is a proof system for \(A\), \(g\) is also a proof system for \(A\). As \(h\) is p-optimal, there is a function \(t \in \FP\) translating \(g\)-proofs to \(h\)-proofs implying that
    \[
      h(t(1w)) = g(1w) = f(g'(w)).
    \]
    This implies \(h'(\langle g'(w), t(1w) \rangle ) = g'(w)\). Hence, \(h'\) p-simulates \(g'\).
  \end{proof}

  In contraposition to this, we can state that for \(B \redmany A\), if \(B\) has no (p-)optimal proof system, then \(A\) has not either.

  \TODO{tell some history of this question \cite[p. 1]{KM00}}The following lemma gives a partial answer to the basic question what languages do have optimal proof systems.

  \begin{lemma}
   \begin{enumerate}[(i)]
    \item Every language in \(\P\) has a p-optimal proof system.
    \item Every language in \(\NP\) has an optimal proof system.
   \end{enumerate}
  \end{lemma}

  \begin{proof}
    \begin{enumerate}[(i)]
      \item 
        Let \(L \in \P\). Then there is a function \(f \in \FP\) with \(f(w) = 1 \Leftrightarrow w \in L\). To show there is a proof system, let \(h\) be defined by
        \[
          h(w) =
          \begin{cases}
            w & (f(w) = 1), \\
            \perp & (\text{otherwise}). \\
          \end{cases}
        \]
        Then \(h\) is a proof system for \(L\). To show \(h\) is optimal, let \(h'\) be an arbitrary proof system for \(L\). Then \(h' \in \FP\) by definition and we can translate \(h'\)-proofs with \(h'\) in polynomial time into \(h\)-proofs, in formulas
        \[
          h(h'(w)) = h'(w).
        \]
        Therefore, \(h\) p-simulates every proof system \(h'\).
      \item
        Let \(L \in \NP\). Then there is a nondeterministic Turing transducer \(M\) deciding \(L\) in polynomial time. Let \(f_i(x) \in \FP\) the function calculating the \(i\)-th path of the nondeterministic calculation of \(M\). Finally, let \(h\) be defined by
        \[
          h(\langle i, w \rangle) =
          \begin{cases}
            f_i(w) & (f_i(w) \text{ accepts}), \\
            \perp & (\text{otherwise}). \\
          \end{cases}
        \]
        Then \(h \in \FP\) is a proof system for \(L\). To show \(h\) is optimal, let \(h'\) be an arbitrary proof system for \(L\). Let \(g\) be a function that maps an \(w \in L\) to an \(i\) such that \(f_i(w)\) accepts in polynomial time. Notice that \(g\) may be not in \(\FP\). With these definitions, we obtain
        \[
          h(\langle g(h'(w)), h'(w) \rangle) = h'(w).
        \]
        Therefore, \(h\) simulates every proof system \(h'\) via the translating function
          \[w \mapsto \langle g(h'(w)), h'(w) \rangle.\]
    \end{enumerate}
  \end{proof}

  This implies \(\NP \subseteq \OPT\). By stating different properties for \(\P\) and \(\NP\), the lemma connects to the \(\P\)-\(\NP\)-question. If one would find a set in \(\NP\) without an p-optimal proof system, one would have separated \(\P\) from \(\NP\).

  \begin{corollary}
    If there is no p-optimal proof system for \(\SAT\), then \(\P \neq \NP\).
  \end{corollary}  
  
  A main motivation to study proof systems is given by the following theorem, as it connections optimal proof systems further to the important question of separating complexity classes \cite{KMT03}.

  \begin{theorem}
   \(\NP = \coNP\) if and only if a polynomial bounded proof system for \(\TAUT\) exists.
  \end{theorem}

  According to Köbler et al., one tried to separate \(\NP\) from \(\coNP\) by studying more and more powerful proof systems, showing that they are not polynomially bounded \cite{KMT03}. \TODO{improve!}

  % results that could be included
  % - theorem 22 BKM09
  % - theorem 1.2 KMT03




  