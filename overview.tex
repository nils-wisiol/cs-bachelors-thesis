\chapter{A brief Overview of Proof Systems} \label{chpOverview}
  After defining the important notions for this thesis, we will give a brief overview of some important results in the field of optimal proof systems.

  One basic lemma that is widely used formalizes a part of the connection between optimal proof systems and polynomial many-one-reducibility. Later in this thesis, we will use it to proof corollary \ref{corHardSets}. The following proof is mainly taken from Köbler et al. \cite{KMT03}.

  \begin{lemma} \label{lemManyOneProofSystem}
    If \(A\) has a (p-)optimal proof system and if \(B \redmany A\), then \(B\) has a (p-)optimal proof system, too.
  \end{lemma}

  \begin{proof}
    Let \(h\) be a p-optimal proof system for \(A\) and let \(B\) many-one reduce to \(A\) via \(f \in \FP\), that is \(x \in B \Leftrightarrow f(x) \in A\). Then \(h'\) defined by
      \[
        h'(\langle x, w \rangle) =
        \begin{cases}
          x & (h(w) = f(x)), \\
          \perp & (\text{otherwise}), \\
        \end{cases}
      \]
    is a proof system for \(B\), as \(h(w) = f(x) \in A\) is equivalent to \(x \in B\). To show \(h'\) is optimal, let \(g'\) be a proof system for \(B\). In order to obtain a proof system for \(A\), let \(g\) be
      \[
        g(bw) =
        \begin{cases}
          h(w) & (b = 0), \\
          f(g'(w)) & (b = 1). \\
        \end{cases}
      \]
    Since both \(h(w)\) and \(f(g'(w))\) are in \(A\) and \(h\) is a proof system for \(A\), \(g\) is also a proof system for \(A\). As \(h\) is p-optimal, there is a function \(t \in \FP\) translating \(g\)-proofs to \(h\)-proofs implying that
    \[
      h(t(1w)) = g(1w) = f(g'(w)).
    \]
    This implies \(h'(\langle g'(w), t(1w) \rangle ) = g'(w)\). Hence, \(h'\) p-simulates \(g'\).
  \end{proof}

  In contraposition to this, we can state that for \(B \redmany A\), if \(B\) has no (p-)optimal proof system, then \(A\) has not either.

  \TODO{tell some history of this question \cite[p. 1]{KM00}}The following lemma gives a partial answer to the basic question what languages do have optimal proof systems.

  \begin{lemma} \label{lemPHasOptimalProofSystem}
   \begin{enumerate}[(i)]
    \item Every language in \(\P\) has a p-optimal proof system.
    \item Every language in \(\NP\) has an optimal proof system.
   \end{enumerate}
  \end{lemma}

  \begin{proof}
    \begin{enumerate}[(i)]
      \item 
        Let \(L \in \P\). Then there is a function \(f \in \FP\) with \(f(w) = 1 \Leftrightarrow w \in L\). To show there is a proof system, let \(h\) be defined by
        \[
          h(w) =
          \begin{cases}
            w & (f(w) = 1), \\
            \perp & (\text{otherwise}). \\
          \end{cases}
        \]
        Then \(h\) is a proof system for \(L\). To show \(h\) is optimal, let \(h'\) be an arbitrary proof system for \(L\). Then \(h' \in \FP\) by definition and we can translate \(h'\)-proofs with \(h'\) in polynomial time into \(h\)-proofs, in formulas
        \[
          h(h'(w)) = h'(w).
        \]
        Therefore, \(h\) p-simulates every proof system \(h'\).
      \item
        Let \(L \in \NP\). Then there is a nondeterministic Turing transducer \(M\) deciding \(L\) in polynomial time. Let \(f_i(x) \in \FP\) the function calculating the \(i\)-th path of the nondeterministic calculation of \(M\). Finally, let \(h\) be defined by
        \[
          h(\langle i, w \rangle) =
          \begin{cases}
            f_i(w) & (f_i(w) \text{ accepts}), \\
            \perp & (\text{otherwise}). \\
          \end{cases}
        \]
        Then \(h \in \FP\) is a proof system for \(L\). To show \(h\) is optimal, let \(h'\) be an arbitrary proof system for \(L\). Let \(g\) be a function that maps an \(w \in L\) to an \(i\) such that \(f_i(w)\) accepts in polynomial time. Notice that \(g\) may be not in \(\FP\). With these definitions, we obtain
        \[
          h(\langle g(h'(w)), h'(w) \rangle) = h'(w).
        \]
        Therefore, \(h\) simulates every proof system \(h'\) via the translating function
          \[w \mapsto \langle g(h'(w)), h'(w) \rangle.\]
    \end{enumerate}
  \end{proof}

  This implies \(\NP \subseteq \OPT\). By stating different properties for \(\P\) and \(\NP\), the lemma connects to the \(\P\)-\(\NP\)-question. If one would find a set in \(\NP\) without an p-optimal proof system, one would have separated \(\P\) from \(\NP\).

  \begin{corollary}
    If there is no p-optimal proof system for \(\SAT\), then \(\P \neq \NP\).
  \end{corollary}  
  
  In order to prove proposition \ref{prpNPcoNP}, we will show two lemma of which the first is taken from the work of Cook and Reckhow \cite{CR79}. It gives a equivalent formulation of the question whether \(\NP = \coNP\).

  \begin{lemma} \label{lemNPisCoNP}
   \(\NP = \coNP\) if and only if \(\TAUT \in \NP\).
  \end{lemma}

  \begin{proof}
    Using a nondeterministic Turing transducer, we can show that an arbitrary formula is not a tautology in polynomial time by guessing and verifying an assignment for which the formula is falsified. Thus, \(\overline{\TAUT} \in \NP\).

    Assume \(\TAUT \in \NP\), and let \(L\) be an arbitrary language with \(L \in \NP\). \TODO{insert part of the proof of theorem 1 from \cite{C71}.} Hence, there is a function \(f \in \FP\) such that \(x \in L \Leftrightarrow f(x) \in \overline{\TAUT}\) respectively \(x \in \overline{L} \Leftrightarrow f(x) \in \TAUT\). Since \(\TAUT \in \NP\), for any given \(x \in \Sigma^*\) we can in nondeterministic polynomial time decide whether \(f(x)\) is in \(\TAUT\). Thus we also can decide in nondeterministic polynomial time whether \(x \in \overline{L}\). It follows that \(\overline{L} \in \NP\). As this proves that \(\NP\) is closed under complement, we obtain \(\NP \subseteq \coNP\). So see that \(\coNP \subseteq \NP\), let an arbitrary language \(\overline{L}\) be in \(\coNP\). By definition we obtain \(L \in \NP\). As \(\NP\) is closed under complement, \(\overline{L} \in \NP\). Thus, \(\coNP \subseteq \NP\).

    Assume \(\TAUT \notin \NP\). As we have seen above, \(\overline{\TAUT} \in \NP\). Hence \(\NP \neq \coNP\).
  \end{proof}

  The second lemma connects with the theory of proof systems by formulating a necessary and sufficient condition for a language being in \(\NP\) \cite{CR79}.

  \begin{lemma} \label{lemNPPolyBoundProofSystem}
    A set \(L \neq \emptyset\) is in \(\NP\) if and only if \(L\) has a polynomially bounded proof system.
  \end{lemma}

  \begin{proof}
   Assume \(L \in \NP\), then some nondeterministic Turing transducer \(M\) accepts \(L\) in polynomial time. Let \(f_i(x) \in \FP\) the function calculating the \(i\)-th path of the nondeterministic calculation of \(M\). We define \(f\) by
   \[
     f(\langle i, x \rangle) =
     \begin{cases}
       x & (f_i(x) \text{ accepts}), \\
       \perp & (\text{otherwise}). \\
     \end{cases}
   \]
   Then \(f\) is a polynomially bounded proof system for \(L\).

   Conversely, assume \(f\) is a polynomially bounded proof system for \(L\). Then a nondeterministic Turing transducer on input \(y\) can guess a proof \(x\) and verify \(f(x) = y\).
  \end{proof}

  Putting lemma \ref{lemNPisCoNP} and \ref{lemNPPolyBoundProofSystem} together, we obtain proposition \ref{prpNPcoNP},
  \[
    \NP = \coNP \Leftrightarrow \text{ there is a polynomially bounded proof system for } \TAUT.
  \]
  Using this theorem, one tried to seperate \(\NP\) from \(\coNP\) by studying more and more powerful proof systems, showing that they are not polynomially bounded \cite{KMT03}. As mentioned before, there was no success on answering this questions until now. To take the notion of optimal proof systems into account, one could ask if there is an optimal or even p-optimal proof system for \(\TAUT\). If that were the case, then the existence of one specific proof system that is not polynomially bounded would suffice to proof that \(\NP \neq \coNP\) and hence \(\P \neq \NP\) \cite{KMT03}.

  Kraj\'{\i}cek and Pudl{\'a}k proved a sufficient condition for the existence of optimal proof systems for \(\TAUT\).

  \begin{theorem}
    If \(\NE = \coNE\) then optimal proof systems for \(\TAUT\) exist. If \(\E = \NE\) then p-optimal proof systems for \(\TAUT\) exists.
  \end{theorem}

  We will omit their proof in this thesis, since it uses many notions of formal logics and a huge equivalence theorem not introduced here.

  With these basic results, we will investigate the question whether there are languages possessing optimal proof systems outside of \(\NP\).

  % results that could be included
  % - theorem 22 BKM09
  % - theorem 1.2 KMT03




  