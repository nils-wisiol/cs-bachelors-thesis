\begin{abstract}
  \begin{otherlanguage}{ngerman}
    Acht Jahre nachdem die \(\P\)-\(\NP\)-Frage von Cook gestellt wurde \cite{C71}, war es auch Cook zusammen mit Reckhow, der den engen Zusammenhang dieser Frage mit Beweissystemen feststellte \cite{CR79}.

    Die Bedeutung der bis heute ungeklärten \(\P\)-\(\NP\)-Frage ist immens. Die Gleichheit dieser Mengen würde bedeuten, dass man zu jedem Problem, dessen Lösung sich leicht überprüfen lässt, auch leicht eine Lösung finden kann. Es wäre beispielsweise genau so schwer, einen Beweis zu verifzieren, wie einen Beweis zu finden. Es würde kaum einen Unterschied machen, ob man die PIN zu einer EC-Karte herausfinden möchte, oder überprüfen möchte, ob die eingegebene Nummer die korrekte ist. Aufgrund dieser unwirklich erscheinenden Implikationen glauben die meisten Komplexitätstheoretiker, dass \(\P \neq \NP\).

    Der Zusammenhang der \(\P\)-\(\NP\)-Frage mit Beweissystemen lässt sich wie folgt formulieren. Existiert kein polynomiell beschränktes Beweissystem für \(\TAUT\), dann folgt \(\P \neq \NP\). Diesen Satz werden wir im Laufe dieser Arbeit beweisen, und anschließend noch etwas tiefer in die Theorie der Beweisssysteme einsteigen.

    Zunächst wird der Leser in die wichtigsten Defintionen und Begriffe aus dem Gebiet der Beweissysteme eingeführt. Anschließend werden in Kapitel \ref{chpOverview} einige wichtige Ergebnisse zusammengefasst. In Kapitel \ref{chpConexpMinusOpt} wird bewiesen, dass in bestimmten superpolynomiellen Komplexitätsklassen Sprachen existieren, die keine optimalen Beweissysteme besitzen. Abschließend werden die Ergebnisse der Arbeit zusammengefasst und ein Ausblick auf weitere interessante und offene Fragen gegeben.
  \end{otherlanguage}
\end{abstract}
