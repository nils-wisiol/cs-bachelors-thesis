\chapter{A set in $\coNEXP \setminus \OPT$}
  In this section, we will show that there are sets without optimal proof systems.
  
  \begin{theorem}
    Let \(t : \IN \to \IN\) be a time-constructible function such that for every polynomial \(p\) there is a number \(n\) with \(p(n) \leq t(n)\). Then there is a language \(L \in \NTIME(t(n))\) that has no p-optimal proof system.
  \end{theorem}

  Jochen Messner showed in \cite{Mes99} that under the same presumptions as in our theorem, there is a language \(L \in \NTIME(t(n))\) without an optimal acceptor. He also proofed that the existence of a optimal acceptor is equivalent to the existence of a optimal proof system for every p-cylinder \(L\).
  
  \begin{proof}
    Let \(f_1, f_2, ...\) be a enumeration of all \(\FP\)-functions with \(\runtime(f_i) \leq n^i + i\). \TODO{How is that obtained?}
    For any \(i > 0\), let \(L_i\) be the regular language described by the expression \(0^i10^*\). Define
      \[ L_i' = \{ x \in L_i | \forall_{ y \in \Sigma^* } |y| \leq t(|x|) \implies f_i(y) \neq x \}. \]
    That is, as long as you put proofs of polynomial length (relative to \(x\)) into \(f_i\), you will not get \(x\) out.
    Let \(L = \bigcup_{i > 0} L_i'\).

    First, we obtain \(L \in \NTIME(t(n))\). For any given \(x\), one can simulate for every \(y \in \Sigma^*\) up to \(t(|x|)\) steps of a deterministic machine calculating \(f_i(y)\).

    For a proof system \(f_i\) with \(f_i(\Sigma^*) = L\), we observe that \(L_i' = L_i\). Assume there is an \(L_i \ni x = 0^i1z \notin L_i'\). Then there is an \(y\) with \(|y| \leq t(|x|)\) and \(f_i(y) = x\). Since \(f_i\) is a proof system for \(L\), this yields \(x = 0^i1z \in L\) and so \(x \in L_i'\), which contradicts the assumption. Therefore, for any \(y\) with \(f_i(y) = x \in L_i\) we know that \(|y| > t(|x|)\). Speaking informally, every proof system \(f_i\) for \(L\) is ``slow'' on \(L_i' \subset L\).

    Assume now, for contradiction, that \(f_i\) is a optimal proof system for \(L\). Let \(g\) be a function defined as
      \[ g(bx) = \begin{cases}
                  f(x) & (b = 0), \\
                  x & (b = 1 \text{ and } x = 0^i10^* \in L_i).
                 \end{cases} \]
    \(g\) is a proof system for \(L\) with polynomial length-bounded proofs for all \(x \in L_i\). As \(f_i\) is optimal, there is a function \(p \in \FP\) such that for all \(x \in L_i'\), \(f_i(p(x)) = g(x)\). But as \(p \in \FP\), \(|p(x)| =: y \leq q(|x|) \leq t(|x|)\) for a polynomial \(q\). According to the definition of \(L_i'\), this yields \(f_i(y) \neq x\). Therefore, \(f_i\) is not optimal on \(L_i'\), which contradicts the assumption that \(f_i\) is optimal on \(L\).
  \end{proof}


