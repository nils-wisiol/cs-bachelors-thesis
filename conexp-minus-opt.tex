\chapter{A set in $\coNEXP \setminus \OPT$} \label{chpConexpMinusOpt}
  In the main theorem of this thesis, we will show that in certain complexity classes above of \(\NP\), there are always languages that possess no optimal proof system. To formalize this, we say a function is \defNotion{time-constructible}, if there is a Turing transducer that stops on input \(n\) after \(t(n)\) steps.

  \section{Basic Construction}
  
  \begin{lemma}\label{lemSimulation}
    \begin{enumerate}[(a)]
     \item There is an enumeration of all \(\FP\)-functions \(f_i\) such that for each function it holds \(\runtime(f_i) \leq n^i + i\).
     \item There is an universal Turing transducer \(U\) that can simulate every program \(\alpha\) in \(O(n^2)\) steps, where \(n\) is the number of steps \(M_\alpha\) runs on the same input.
     \item There is an universal Turing transducer \(U\) that can for every \(i \in \IN\) and every input \(y \in \Sigma^*\) simulate \(f_i(y)\) in \(O((n^i + i)^2) = O(n^{2i})\) steps, where \(n = |y|\).
    \end{enumerate}
  \end{lemma}
  
  
  \begin{theorem}\label{thmMain}
    Let \(t : \IN \to \IN\) be a time-constructible function such that for every polynomial \(p\) there is a number \(n\) with \(p(n) \leq t(n)\). Then there is a language \(L \in \coNTIME(t(n))\) that has no optimal proof system.
  \end{theorem}

  Notice, if for function \(f\) there is an \(n \in \IN\) with \(f(n) \geq p(n)\) for every polynomial \(p\), then for every polynomial \(p\) there are infinitely many \(n \in \IN\) for which \(f(n) \geq p(n)\) is satisfied. Otherwise, we could construct a polynomial \(q\) that is greater than \(f\) for every \(n \in \IN\). To see this, let \(p\) be a polynomial for which \(f(n) \geq p(n)\) for only finitely many \(n \in \IN\). Then we can determine the greatest difference between \(f\) and \(p\) by taking the maximum of the set \(\eta = \{ f(n) - p(n), f(n) \geq p(n) \}\). Then for the polynomial \(p'(n) = \eta + p(n) + 1\) there is no \(n\) with \(f(n) \geq p'(n)\), which contradicts the assumption that for every polynomial \(p\) there is a natural number such that \(f(n) \geq p(n)\).

  Messner showed that under the same presumptions as in our theorem, there is a language \(L \in \coNTIME(t(n))\) without an optimal acceptor \cite{Mes99}. He also proved that the existence of an optimal acceptor is equivalent to the existence of an optimal proof system for every p-cylinder \(L\). We will here give a proof that is based on the work of Messner, but stays in the notion of proof systems.

  Before formally proofing theorem \ref{thmMain}, let us briefly discuss the main idea of the proof. As a first observation, we notice that we can enumerate all \(\FP\)-functions \(f_i\) such that \(\runtime(f_i) \leq n^i + i\). We then construct a language \(L\) by uniting languages \(L_i\). We will construct them in a way such that there are only long \(f_i\)-proofs for all strings in \(L_i\). Afterwards we assume there is an optimal proof system \(g\) for \(L\). As for every \(\FP\)-function there is a subset that has only long proofs, there is such a subset \(L_g\) for \(g\). But as it turns out we can state a proof system for \(L_g\) that has short proofs, we can combine both proof systems to a system that runs polynomial equally fast on nearly every string in \(L\), but way faster on strings in \(L_g\). Hence, \(g\) cannot be an optimal proof system.
  
  \begin{proof}[Formal proof of theorem \ref{thmMain}]
    Let \(f_1, f_2, ...\) be a enumeration of all \(\FP\)-functions with \(\runtime(f_i) \leq n^i + i\).
    We obtain this enumeration by enumerating all Turing transducers (using Gödel's enumeration) and combine each transducer with a clock that ensures a maximum runtime of \(n^i + i\) steps for a transducer \(M_i\).
    For any \(i > 0\), let \(L_i\) be the regular language described by the expression \(0^i10^*\). Define
      \[ L_i' = \{ x \in L_i | \forall_{ y \in \Sigma^* } |y|^{2i} \leq t(|x|) \implies f_i(y) \neq x \}. \]
    That is, as long as you put strings of length \(|y|^{2i} \leq t(|x|)\) into \(f_i\), you will not obtain \(x\).
    Let \(L = \bigcup_{i > 0} L_i'\).

    First, we obtain \(L \in \coNTIME(t(n))\). To show this, one considers
      \[L \in \coNTIME(t(n)) \Leftrightarrow \overline{L} = \overline{\bigcup_{i > 0} L_i'} = \bigcap_{i > 0} \overline{L_i'} \in \NTIME(t(n)).\]
    By negating the condition for \(L_i'\), we obtain
      \[ \overline{L_i'} = \{ x \in \Sigma^* | x \notin L_i \vee \left( \exists_{y \in \Sigma^*} |y|^{2i} \leq t(|x|) \wedge f_i(y) = x \right) \}. \]
    For any given \(x\), we can decide in polynomial time whether it is in any \(L_i\) or not. If it is not, then \(x\) is in \(\overline{L_i'}\) for all \(i > 0\) and therefore \(x \in \overline{L}\), so we are done. If it is in any \(L_i\), it is in exactly one \(L_i\). Let \(i^*\) be the set with \(x \in L_{i^*}\). We can simulate \TODO{have a look at simulation runtime} a polynomial-time machine calculating \(f_{i^*}(y)\) on every input \(y \in \Sigma^*\) with \(|y|^{2i} \leq t(|x|)\). If, and only if, there is a path with \(f_{i^*}(y) = x\), then \(x \in \overline{L}\). In both cases, \(\overline{L} \in \NTIME(t(n))\).
    \TODO{more explicit}

    For a proof system \(f_i\) with \(f_i(\Sigma^*) = L\), we observe that \(L_i' = L_i\). Assume there is an \(x = 0^i1z \in L_i\) that is not in \(L_i'\). Then there is an \(y\) with \(|y|^{2i} \leq t(|x|)\) and \(f_i(y) = x\). Since \(f_i\) is a proof system for \(L\), this yields \(x = 0^i1z \in L\) and so \(x \in L_i'\), which contradicts the assumption. Therefore, for any \(y\) with \(f_i(y) = x \in L_i\) we know that \(|y|^{2i} > t(|x|)\). Speaking informally, every proof system \(f_i\) for \(L\) has long proofs on \(L_i' \subset L\).

    Assume now, for contradiction, that \(f_i\) is an optimal proof system for \(L\). Let \(g\) be a function defined as
      \[ g(bx) = \begin{cases}
                  f_i(x) & (b = 0), \\
                  x & (b = 1 \text{ and } x = 0^i10^* \in L_i = L_i').
                 \end{cases} \]
    \(g\) is a proof system for \(L\) with polynomial length-bounded proofs for all \(x \in L_i\). As \(f_i\) is optimal, there is a function \(f^*\) such that for all \(x \in L_i'\), \(f_i(f^*(x)) = g(x)\) and \(|f^*(x)| \leq p(|x|)\) for a polynomial \(p\). Let \(q\) be the polynomial \(q(n) = p(n)^{2i}\). Then \(p(|x|) \leq p(|x|)^{2i}\). As there is an \(n\) with \(q(n) \leq t(n)\), there is an \(x\) in \(L_i\) such that \(|f^*(x)| \leq p(|x|) \leq q(|x|) = p(|x|)^{2i} \leq t(|x|)\). According to the definition of \(L_i'\), this yields \(f_i(f^*(x)) \neq x\). Therefore, \(f_i\) is not optimal on \(L_i'\), which contradicts the assumption that \(f_i\) is optimal on \(L\).
  \end{proof}

  Using the relation to many-one-hard reducible sets proven in lemma \ref{lemManyOneProofSystem}, we obtain

  \begin{corollary} \label{corHardSets}
    No set \(\redmany\)-hard for \(\coNE\) has an optimal proof system.
  \end{corollary}

  \begin{proof}
    With \(t(n) = 2^n\), we can get an \(L \in \coNE\) that has no optimal proof system. Any \(\redmany\)-hard set \(A\) for \(\coNE\) is \(L \redmany A\). Together with lemma \ref{lemManyOneProofSystem} we obtain, that \(A\) cannot have an optimal proof system.
  \end{proof}

  \section{Making the Language Tally}

  Now, let us take a closer look at this set \(L\) that has no optimal proof system. One first observation is that \(L\) is sparse. As every \(L_i'\) only contains strings that are of the form \(0^i10^*\), \(L\) is a subset of the regular language \(L_R = 0^*10^*\). Therefore, the density of \(L_R\) is an upper bound for the density of \(L\). In \(L_R\), there are exactly \(n\) words of length \(n\). As a consequence, \(\dens_{L_R}(n) = \sum_{i=1}^n i = \frac{n^2 + n}{2}\). Hence, \(L_R\) and \(L\) are both sparse. But we can construct even less dense languages by constructing tally sets without optimal proof systems.

  \begin{theorem} \label{thmTally}
    If \(L \subseteq 0^*10^* \) has no optimal proof system, then there is a \(T \in \TALLY\), that has no optimal proof system and is polynomial time equivalent to \(L\).
  \end{theorem}

  \begin{proof}
    Using the Cantor pairing function, we can construct a bijective function \(t \in \FP\) thats maps each element of \(L\) one-to-one to an element of \(T \subseteq \TALLY\). Since all strings in \(L\) are almost tally---namely every string in \(L\) contains only one 1---\(t\) is a polynomial time function. Let \(g\) be an optimal proof system for \(T = \{ t(x), x \in L \}\). Then \(t^{-1} \circ g\) is a proof system for \(L\). Let \(h\) be an arbitrary proof system for \(L\). Then \(t \circ h\) is a proof system for \(T\). As \(g\) is optimal, there is a polynomial bounded \(f\) with \(g(f(x)) = t(h(x))\). It follows that \(t^{-1}(g(f(x))) = h(x)\) for all proof systems \(h\) for \(L\). Hence, \(t^{-1} \circ g\) is an optimal proof system for \(L\), which contradicts the assumption that \(L\) has no optimal proof system. Using \(t\) we obtain \(L \redmany T\) and \(T \redmany L\).
  \end{proof}
  
  Using this theorem, we can further improve our results on the density of languages not possessing an optimal proof system. For any tally language \(p(n) = n\) is an upper bound of its density.

  \begin{corollary}
    There is a language \(L\) with \(\dens_L(n) \leq n\) that possesses no optimal proof system.
  \end{corollary}

  \section{Super-Sparse Sets without an Optimal Proof System}

  By adapting the proof of our main theorem \ref{thmMain}, we can show that there are even arbitrary sparse sets in \(\coNTIME(t(n))\) that do not possess an optimal proof system. In order to do so, we have to slightly tighten the requirements of theorem \ref{thmMain}. To formalize the notion of arbitrary sparse, we introduce a polynomial-time computable, strictly increasing function \(u : \IN \to \IN\). Notice, as \(u\) is strictly increasing, we have \(u(n) \geq n\) for all \(n \in \IN\).

  We will use \(u\) to constrain the length of any string in \(L\). Therefore, we define
  \[
    \widetilde{L}'_i = \{ x \in L'_i : \exists_{k \in \IN} |x| = u(k) \}. 
  \]
  By definition we obtain \(\widetilde{L}'_i \subseteq L_i\). We further define \(\widetilde{L} = \bigcup_{i>0} \widetilde{L}_i\). By doing so, in \(\widetilde{L}\) will only be strings whose length is in the range of \(u\). Informally speaking, the faster \(u\) is growing, the more sparse becomes \(\widetilde{L}\). More formally, we can state an upper bound for the density of \(\widetilde{L}\) as follows.
  \begin{eqnarray*}
    \dens_{\widetilde{L}}(n) 
    &=& \sum_{k=0}^n \sharp \{ x \in \widetilde{L} : |x| = k \} \\
    &\leq& \sum_{k=0}^n \sharp \{ x \in \Sigma^* : x = 0^a10^b \text{ and } |x| = k \} \\
  \end{eqnarray*}
  The faster \(u\) is growing, the more summands in this sum are zero, as there are many \(k\) not in the range of \(u\). The exact count of summands is \(u^{-1}(n)\), which we will define as the the greatest number \(x\) for which \(u(x) \leq n\). For each summand, \(k\) is an upper bound of its value, and as \(k \leq n\) we obtain 
  \[
   \dens_{\widetilde{L}}(n) \leq u^{-1}(n) \cdot n.
  \]
  \TODO{give an example}

  In order to adapt the proof of theorem \ref{thmMain} to this new circumstances, we have to adapt one issue. It occurs in the proof that \(\widetilde{L}\) is in \(\coNTIME(t(n))\). In order to decide whether an given string \(x\) is in the complement \(\widetilde{L}\), we have to additionally check in polynomial time whether there is an \(k \in \IN\) such that \(|x| = u(k)\). Since \(u \in \FP\) and is strictly increasing, this is possible. As for any \(k > |x|\), it follows \(u(k) \geq |x|\), we have to calculate at most \(|x|\) times the value of \(u(k)\) in order to check if there is an \(k\) with \(u(k) = |x|\). If there is no such \(k\), then the given \(x\) is for sure in the complement of \(\widetilde{L}\) and we are done. If there is such a \(k\), we proceed with the algorithm as described before.

  Adapting the proof as described yields
  
  \begin{corollary} \label{corSparse}
    Let \(t:\IN\to\IN\) be a time-constructible function such that for every polynomial \(p\) there is a number \(n\) with \(p(n) \leq t(n)\). Let \(u\) be a polynomial-time computable, strictly increasing function \(u:\IN\to\IN\). Then there is a language \(L \in \coNTIME(t(n))\) that has no optimal proof system; the density of \(L\) is bounded by 
    \[
     \dens_L(n) \leq u^{-1}(n) \cdot n,
    \]
    where \(u^{-1}\) denotes the smallest number \(m\) for which \(u(m) \geq n\).
  \end{corollary}


  \section{Adding Redundancy}

  Given a language \(L\) not possessing an optimal proof system, we can show that there is even a mitotic language that has no optimal proof system. To show this, we mainly rely on results proven in chapter \ref{chpOverview}.

  \begin{theorem} \label{thmMitotic}
    If \(L\) possesses no optimal proof system, there is an polynomial time equivalent, many-one-mitotic language \(L' = 0L \cup 1L\) not possessing an optimal proof system, too.
  \end{theorem}

  \begin{proof}
    As \(L\) has no optimal proof system, corollary \ref{corFurtherOPT} states that \(L' = 0L \cup 1L\) is polynomial time equivalent and does not possess an optimal proof system.

    Let \(A\) be the language \(0\Sigma^*\), then \(\overline{A} = 1\Sigma^*\). We obtain \(L' \cap A = 0L\) and \(L' \cap \overline{A} = 1L\). As shown in corollary \ref{corFurtherOPT}, \(L\) and \(bL\) are polynomial time equivalent. Using \(0x \mapsto 1x\), \(0L\) and \(1L\) are also polynomial time equivalent. Hence, \(L'\) is many-one-mitotic.
  \end{proof}
  
  
  \section{Putting the Results Together}
  
  In this section, we will show that we can combine all results to one language, that is arbitrary sparse, tally, has no optimal proof system and possesses redundancy.
  
  \begin{corollary}
    Let \(t:\IN\to\IN\) be a time-constructible function such that for every polynomial \(p\) there is a number \(n\) with \(p(n) \leq t(n)\). Let \(u\) be a polynomial-time computable, strictly increasing function \(u:\IN\to\IN\). Then there is a language \(L \in \coNTIME(t(n))\) that possesses no optimal proof system and 
    \begin{enumerate}[(i)]
     \item has its density bound by \(\dens_L(n) \leq u^{-1}(n) \cdot n\), where \(u^{-1}\) denotes the smallest number \(m\) for which \(u(m) \geq n\),
     \item is many-one-mitotic,
     \item is tally.
    \end{enumerate}
  \end{corollary}
  
  \begin{proof}
    Corollary \ref{corSparse} yields a language \(L \in \coNTIME(t(n))\) that has no optimal proof system and fulfills our demands on its sparseness. By using theorem \ref{thmMitotic}, we obtain the language \(0L \cup 1L\) that is polynomial time-equivalent to \(L\) and possesses no optimal proof system. Finally, we can use theorem \ref{thmTally} to construct a tally set \(T \notin \OPT\) and is polynomial time-equivalent to \(L\). \TODO{\(T \in \coNTIME(t(n))\)?}
  \end{proof}



 

  