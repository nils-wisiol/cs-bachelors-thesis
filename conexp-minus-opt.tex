\chapter{A set in $\coNEXP \setminus \OPT$} \label{chpConexpMinusOpt}
  In the main theorem of this thesis, we will show that in certain complexity classes above of \(\NP\), there are always languages that possess no optimal proof system. To formalize this, we say a function is \defNotion{time-constructible}, if there is a Turing transducer that stops on input \(n\) after \(t(n)\) steps.
  
  \begin{theorem}\label{thmMain}
    Let \(t : \IN \to \IN\) be a time-constructible function such that for every polynomial \(p\) there is a number \(n\) with \(p(n) \leq t(n)\). Then there is a language \(L \in \coNTIME(t(n))\) that has no optimal proof system.
  \end{theorem}

  Notice, if for function \(f\) there is an \(n \in \IN\) with \(f(n) \geq p(n)\) for every polynomial \(p\), then there are for every polynomial \(p\) infinitely many \(n \in \IN\) for which \(f(n) \geq p(n)\) is satisfied. Otherwise, we could construct a polynomial \(q\) that is greater than \(f\) for every \(n \in \IN\). To see this, let \(p\) be a polynomial for which \(f(n) \geq p(n)\) for only finitely many \(n \in \IN\). Then we can determine the greatest difference between \(f\) and \(p\) by taking the maximum of the set \(\eta = \{ f(n) - p(n), f(n) \geq p(n) \}\). Then for the polynomial \(p'(n) = \eta + p(n) + 1\) there is no \(n\) with \(f(n) \geq p'(n)\), which contradicts the assumption that for every polynomial \(p\) there is a natural number such that \(f(n) \geq p(n)\).

  Messner showed that under the same presumptions as in our theorem, there is a language \(L \in \coNTIME(t(n))\) without an optimal acceptor \cite{Mes99}. He also proofed that the existence of a optimal acceptor is equivalent to the existence of a optimal proof system for every p-cylinder \(L\). We will here give a proof that is based on the work of Messner, but stays in the notion of proof systems.

  Before formally proofing theorem \ref{thmMain}, let us briefly discuss the main idea of the proof. As a first observation, we notice that we can enumerate all \(\FP\)-functions \(f_i\) such that \(\runtime(f_i) \leq n^i + i\). We then construct a language \(L\) by uniting languages \(L_i\). We will construct them in a way such that there are only long \(f_i\)-proofs for all strings in \(L_i\). Afterwards we assume there is an optimal proof system \(g\) for \(L\). As for every \(\FP\)-function there is a subset that has only long proofs, there is such a subset \(L_g\) for \(g\). But as it turns out we can state a proof system for \(L_g\) that has short proofs, we can combine both proof systems to a system that runs polynomial equally fast on nearly every string in \(L\), but way faster on strings in \(L_g\). Hence, \(g\) cannot be a optimal proof system.
  
  \begin{proof}[Formal proof of theorem \ref{thmMain}]
    Let \(f_1, f_2, ...\) be a enumeration of all \(\FP\)-functions with \(\runtime(f_i) \leq n^i + i\). \TODO{How is that obtained?}
    For any \(i > 0\), let \(L_i\) be the regular language described by the expression \(0^i10^*\). Define
      \[ L_i' = \{ x \in L_i | \forall_{ y \in \Sigma^* } |y|^{2i} \leq t(|x|) \implies f_i(y) \neq x \}. \]
    That is, as long as you put strings of length \(|y|^{2i} \leq t(|x|)\) into \(f_i\), you will not obtain \(x\).
    Let \(L = \bigcup_{i > 0} L_i'\).

    First, we obtain \(L \in \coNTIME(t(n))\). To show this, one considers
      \[L \in \coNTIME \Leftrightarrow \overline{L} = \overline{\bigcup_{i > 0} L_i'} = \bigcap_{i > 0} \overline{L_i'} \in \NTIME.\]
    By negating the condition for \(L_i'\), we get
      \[ \overline{L_i'} = \{ x \in \Sigma^* | x \notin L_i \vee \left( \exists_{y \in \Sigma^*} |y| \leq t(|x|) \wedge f_i(y) = x \right) \}. \]
    For any given \(x\), we can decide in polynomial time whether it is in any \(L_i\) or not. If it is not, then \(x\) is in \(\overline{L_i'}\) for all \(i > 0\) and therefore \(x \in \overline{L}\), so we are done. If it is in any \(L_i\), it is in exactly one \(L_i\). Let \(i^*\) be the set with \(x \in L_{i^*}\). We can simulate \TODO{have a look at simulation runtime} a polynomial-time machine calculating \(f_{i^*}(y)\) on every input \(y \in \Sigma^*\) with \(|y|^{2i} \leq t(|x|)\). If, and only if, there is a path with \(f_{i^*}(y) = x\), then \(x \in \overline{L}\). In both cases, \(\overline{L} \in \NTIME(t(n))\).

    For a proof system \(f_i\) with \(f_i(\Sigma^*) = L\), we observe that \(L_i' = L_i\). Assume there is an \(x = 0^i1z \in L_i\) that is not in \(L_i'\). Then there is an \(y\) with \(|y|^{2i} \leq t(|x|)\) and \(f_i(y) = x\). Since \(f_i\) is a proof system for \(L\), this yields \(x = 0^i1z \in L\) and so \(x \in L_i'\), which contradicts the assumption. Therefore, for any \(y\) with \(f_i(y) = x \in L_i\) we know that \(|y|^{2i} > t(|x|)\). Speaking informally, every proof system \(f_i\) for \(L\) has long proofs on \(L_i' \subset L\).

    Assume now, for contradiction, that \(f_i\) is a optimal proof system for \(L\). Let \(g\) be a function defined as
      \[ g(bx) = \begin{cases}
                  f_i(x) & (b = 0), \\
                  x & (b = 1 \text{ and } x = 0^i10^* \in L_i = L_i').
                 \end{cases} \]
    \(g\) is a proof system for \(L\) with polynomial length-bounded proofs for all \(x \in L_i\). As \(f_i\) is optimal, there is a function \(f^*\) such that for all \(x \in L_i'\), \(f_i(f^*(x)) = g(x)\) and \(|f^*(x)| \leq p(|x|)\) for a polynomial \(p\). Let \(q\) be the polynomial \(q(n) = p(n)^{2i}\). As \(p(|x|)\) is positive, \(p(|x|) \leq p(|x|)^{2i}\). As there is an \(n\) with \(q(n) \leq t(n)\), there is an \(x\) in \(L_i\) such that \(|f^*(x)| \leq p(|x|) \leq q(|x|) = p(|x|)^{2i} \leq t(|x|)\). According to the definition of \(L_i'\), this yields \(f_i(f^*(x)) \neq x\). Therefore, \(f_i\) is not optimal on \(L_i'\), which contradicts the assumption that \(f_i\) is optimal on \(L\).
  \end{proof}

  Using the relation to many-one-hard reducible sets proofed in lemma \ref{lemManyOneProofSystem}, we obtain

  \begin{corollary} \label{corHardSets}
    No set \(\redmany\)-hard for \(\coNE\) has an optimal proof system. \TODO{Is this possible for \(\coNEXP\)?}
  \end{corollary}

  \begin{proof}
    With \(t(n) = 2^n\), we can get an \(L \in \coNE\) that has no optimal proof system. Any \(\redmany\)-hard set \(A\) for \(\coNE\) is \(L \redmany A\). Together with lemma \ref{lemManyOneProofSystem} we obtain, that \(A\) cannot have optimal proof system.
  \end{proof}

  Now, let us take a closer look at this set \(L\) that has no optimal proof system. One first observation is that \(L\) is sparse. As every \(L_i'\) only contains strings that are of the form \(0^i10^*\), \(L\) is a subset of the regular language \(L_R = 0^*10^*\). Therefore, the density of \(L_R\) is an upper bound for the density of \(L\). In \(L_R\), there are exactly \(n\) words of length \(n\). As a consequence, \(\dens_{L_R}(n) = \sum_{i=1}^n i = \frac{n^2 + n}{2}\). Hence, \(L_R\) and \(L\) are both sparse.

  But, how sparse can \(L\) be, maintaining its property of not possessing a optimal proof system? We can make \(L\) even more sparse by intersecting with sparse sets. For that purpose, let \(I_k^i\) be defined by \(I_k^i = \{ x \in \Sigma^*, x = 0^*1(0^k)^i \}\). For a fixed length \(j\), in \(I_k^i\) there is exactly one string, namely \(0^{j-1-ki}1(0^k)^i\) if \(j \geq 1 + ki\). An equivalent description of this constraint is given by
  \[
    j \geq 1 + ki \Leftrightarrow
    \frac{j-1}{k} \geq i \Leftrightarrow
    \left \lfloor \frac{j-1}{k} \right \rfloor \geq i.
  \]
  Otherwise, there is no word of this length. Hence, the language \(I_k\) defined by \(I_k = \bigcup_{i \in \IN} I_k^i\) contains exactly \(\lfloor \frac{j-1}{k} \rfloor\) strings of length exactly \(j\). Notice \(I_k\) is the regular language \(I_k = 0^*1(0^k)^*\). We obtain an upper bound for the density of \(I_k\) as follows
  \[
    \dens_{I_k}(n) = \sum_{j=1}^{n} \left \lfloor \frac{j-1}{k} \right \rfloor \leq \frac{1}{k} \sum_{j=1}^n j-1 = \frac{(n-1)^2-n+1}{2k}.
  \]
  \TODO{improve this section}

  We also can provide an tally language that has no optimal proof system, since we can convert from \(\Sigma^*\) to \(\TALLY\) in polynomial time.

  \begin{corollary}
    If \(L \in \C\) has no optimal proof system, then there is a \(T \in \C \cap \TALLY\) that has no optimal proof system.
  \end{corollary}

  \begin{proof}
    Let \(t\) be a bijective function \(\Sigma^* \to \TALLY\) with \(t \in \FP\). \TODO{there is no function like that} Let \(g\) be an optimal proof system for \(T = \{ t(x), x \in L \}\). Then \(t^{-1} \circ g\) is a proof system for \(L\). Let \(h\) be an arbitrary proof system for \(L\). Then \(t \circ h\) is a proof system for \(T\). As \(g\) is optimal, there is a \(f \in \FP\) with \(g(f(x)) = t(h(x))\). It follows that \(t^{-1}(g(f(x))) = h(x)\) for all proof systems \(h\) for \(L\). Hence, \(t^{-1} \circ g\) is a optimal proof system for \(L\), which contradicts the assumption.
  \end{proof}


  



