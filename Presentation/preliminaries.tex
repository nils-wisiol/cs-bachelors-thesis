\section{Preliminaries} 
\subsection{Optimal Proof Systems}

\begin{frame}
  \frametitle{Optimal Proof Systems}
  
  \onslide<1-3> A \(\FP\)-function \(h\) is called \defNotion{proof system} for a language \(L\), if \(f(\Sigma^*) = L\).
  \onslide<2-3> If \(h(w) = x\) holds, we call \(w\) a \defNotion{\(h\)-proof} for \(x\).
  %\onslide<3-3> \(h\) ist \defNotion{polynomiell beschränkt}, wenn es ein Polynom \(p\) gibt, so dass für jedes \(x \in L\) ein \(h\)-Beweis \(w\) mit \(|w| \leq p(|x|)\) existiert. 
  
  \onslide<3-7> Let \(h\) and \(h'\) be proof systems for a language \(L\),
  \onslide<4-7> and let \(p\) be a polynomial and a let \(f\) be a function
  \onslide<5-7> such that for all \(h'\)-proofs \(w\) it holds that \[ h(f(w)) = h'(w), \]
  \onslide<6-7> where \(|f(w)| \leq p(|w|)\) holds.
  \onslide<7-7> In this case, we say \(h\) \defNotion{simulates} the proof system \(h'\).
  
  \onslide<8-9> If a proof system \(h\) for a language \(L\) simulates every other proof system for the same language \(L\), we call it \defNotion{optimal}.
  \onslide<9-9> Let \(\OPT\) be the complexity class of all languages possessing a optimal proof system.
\end{frame}

