\section{Folgerungen} 

\begin{frame}
  \frametitle{Folgerungen}

  \pause

  Welche Struktur besitzt \(L\)?
  
  \pause
  
  \begin{theorem}
    Besitzt \(L \subseteq 0^*10^*\) kein optimales Beweissystem, dann gibt es ein polynomialzeitäquivalentes \(T \in \TALLY\), das ebenfalls kein optimales Beweissystem besitzt.
  \end{theorem}

  \pause
  
  \begin{corollary}
    Sei \(u : \IN \to \IN\) eine Polynomialzeitfunktion, die monoton wächst. Dann gibt es eine Sprache \(L \in \coNTIME(2^n)\), die kein optimales Beweissystem besitzt und nur Wörter der Länge \(u(\IN)\) enthält.
  \end{corollary}

  \pause

  \begin{theorem}
    Besitzt \(L\) kein optimales Beweissystem, dann gibt es eine polynomialzeitäquivalente, many-one-mitotische Sprache, die ebenfalls kein Beweissystem besitzt.
  \end{theorem}



\end{frame}
