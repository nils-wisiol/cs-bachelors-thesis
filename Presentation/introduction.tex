\section{Einführung} 
\subsection{Beweissysteme und die $\P$-$\NP$-Frage}

\begin{frame}
  \frametitle{Die \(\P\)-\(\NP\)-Frage}
  
  \begin{quotation}
    ``If \(\P=\NP\), then the world would be a profoundly different place than we usually assume it to be. There would be no special value in `creative leaps', no fundamental gap between solving a problem and recognizing the solution once it's found. Everyone who could appreciate a symphony would be Mozart; everyone who could follow a step-by-step argument would be Gauss; everyone who could recognize a good investment strategy would be Warren Buffett.``
  \end{quotation}
  
  \rciteplain{Scott Aaronson \\ \footnotesize{\href{http://www.scottaaronson.com/}{scottaaronson.com}}}
   
\end{frame}

\begin{frame}
  \frametitle{Konsequenzen aus der Theorie der Beweissysteme für die \(\P\)-\(\NP\)-Frage}
  
  \begin{lemma}
    \(\NP \neq \coNP \implies \P \neq \NP\).
  \end{lemma}  
 
  \pause
  
  \begin{proposition} \label{prpNPcoNP}
    Es ist genau dann \(\NP\) = \(\coNP\), wenn ein polynomiell beschränktes Beweissystem für \(\TAUT\) existiert.
    \rcite{KMT03}
  \end{proposition}
  
\end{frame}
