\chapter{Introduction}
  After Cook stated 1971 the \(\P\) versus \(\NP\) question \cite{C71}, he gave the main motivation to study proof systems in 1979. Cook and Reckhow showed in their article the close relation between the separation of complexity classes and the existence of polynomially bounded proof systems \cite{CR79}.

  Despite its importance, the \(\P\) versus \(\NP\) question remains still open. Informally speaking, \(\P = \NP\) means that for every problem that has an efficiently verifiable solution, we can find that solution efficiently as well. While most theoreticians assume that \(\P \neq \NP\), their equality would have heavy implications. One consequence affects the security of communication in the Internet, as public-key cryptography depends on the existence of certain problems in \(\NP\) that are not efficiently to decide. If there are no problems with solutions fast to verify but hard to find, as \(\P = \NP\) would imply, the small lock beside the URL in one's browser could not indicate a secure communication anymore \cite{F09}.

  \(\P = \NP\) would also have fundamental implications on mathematics: As mathematical proofs must by definition be efficiently verifiable, \(\P = \NP\) would imply that they are efficiently to find \cite{CR79}. This argument to believe that \(\P \neq \NP\) is further illustrated in the following quotation, taken from Aaronsons Blog\footnote{\url{http://www.scottaaronson.com/blog/}}.

  \begin{quotation}
    ``If \(\P=\NP\), then the world would be a profoundly different place than we usually assume it to be. There would be no special value in `creative leaps', no fundamental gap between solving a problem and recognizing the solution once it's found. Everyone who could appreciate a symphony would be Mozart; everyone who could follow a step-by-step argument would be Gauss; everyone who could recognize a good investment strategy would be Warren Buffett.``
  \end{quotation}

  Closely related to the \(\P\) versus \(\NP\) problem is the question if \(\NP = \coNP\). If \(\P = \NP\), then \(\NP = \coNP\), since \(\P\) is closed under complement. In return, if one can separate \(\NP\) from \(\coNP\), then \(\P \neq \NP\). To connect the field of proof systems with the \(\P\) versus \(\NP\) questions, we state

  \begin{proposition}[\cite{KMT03}, \cite{CR79}] \label{prpNPcoNP}
    \(\NP\) = \(\coNP\) if and only if a polynomially bounded proof system for \(\TAUT\) exists.
  \end{proposition}

  In order to introduce the reader into the field of proof systems, we will first define the important notions used related to it. Subsequently, we will prove proposition \ref{prpNPcoNP} and give an overview of important results about proof systems in chapter \ref{chpOverview}. These results will connect proof systems with the \(\P\) versus \(\NP\) question as mentioned above and will give a basis for later proofs.
  In chapter \ref{chpConexpMinusOpt} we will proof the main theorem of this thesis, showing that there are languages without optimal proof systems in all super-polynomial complexity classes. This chapter will also analyze the inner structure of these languages that do not possess an optimal proof system.

  Finally, we will give a conclusion and look forward to currently unresolved problems and further questions.

