\chapter{Introduction}
  After Cook asked 1971 the \(\P\) versus \(\NP\) question \cite{C71}, he gave the main motivation to study proof systems in 1979. Reckhow and Cook showed in their article the close relation between the separation of complexity classes and the existence of polynomially bounded proof systems \cite{CR79}.

  Despite its importance, at the time of writing, the \(\P\) versus \(\NP\) questions remains still open. While most theoreticians assume that \(\P \neq \NP\), their equality would have many implications. Informally speaking, \(\P = \NP\) means that for every problem that has a efficiently verifiable solution, we can find that solution efficiently as well. As a consequence, public-key cryptography, the ability to send secure messages without privately exchanging keys, would be impossible. The small lock in one's browser beside the URL could not indicate a secure connection anymore \cite{F09}.

  \(\P = \NP\) would also have fundamental implications on mathematics: By using a computer, one could determine efficiently whether a given proposition has a proof of a certain length within a given theory \cite{CR79}.

  The question if \(\NP = \coNP\) is closely related to the \(\P\) versus \(\NP\) problem. If \(\P = \NP\), then \(\NP = \coNP\), since \(\P\) is closed under complement. In return, if one can separate \(\NP\) from \(\coNP\), then \(\P \neq \NP\). To connect the field of proof systems with the \(\P\) versus \(\NP\) questions, we state

  \begin{proposition}[\cite{KMT03}, \cite{CR79}] \label{prpNPcoNP}
    \(\NP\) = \(\coNP\) if and only if a polynomially bounded proof systems for \(\TAUT\) exists.
  \end{proposition}

  To get insight into the field of proof systems, we will first define the notions used in the following chapter. Subsequently, we will give an overview of important results about proof systems in chapter \ref{chpOverview}. In this chapter, we will also proof proposition \ref{prpNPcoNP}. After this, we will show the existence of languages without optimal proof systems in certain complexity classes. Finally, we will give a conclusion and look forward to currently unresolved problems and further questions.

