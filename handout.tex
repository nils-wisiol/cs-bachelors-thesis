\documentclass[letterpaper,parskip]{scrartcl}
\usepackage[utf8x]{inputenc}
\usepackage[left=40mm,right=40mm]{geometry}

% Konfiguration
% languages
\usepackage[ngerman,english]{babel}

% Math related packages
\usepackage{amsmath,amsthm,amssymb}
\usepackage{mathtools}

% typesetting (layout)
\usepackage{array}

% graphics
\usepackage{graphicx}

% Enumerations
\usepackage{enumerate}

% markup for text
\newcommand{\defNotion}[1]{\emph{#1}} % for the defined notion in a definition

% symbols
%% sets of numbers
\newcommand{\IN}{\mathbb{N}}
%% sets of functions
\newcommand{\setOfFunctions}[1]{\mathcal{#1}}
\newcommand{\FP}{\setOfFunctions{FP}}
%% complexity classes
\newcommand{\cclass}[1]{\text{#1}}
\newcommand{\OPT}{\cclass{OPT}}
\newcommand{\NTIME}{\cclass{NTIME}}
\newcommand{\coNTIME}{\cclass{co-NTIME}}
\newcommand{\NEXP}{\cclass{co-NEXP}}
\newcommand{\coNEXP}{\cclass{co-NEXP}}

% operators
%% on machines
\DeclareMathOperator{\runtime}{time}

\newtheorem{definition}{Definition}
\newtheorem{theorem}{Theorem}
\newtheorem{lemma}[theorem]{Lemma} % Lemma numbering together with theorem
\newtheorem{corollary}[theorem]{Corollary} % Corollary numbering together with theorem
\newtheorem{proposition}[theorem]{Proposition} % Proposition numbering together with theorem



\begin{document}

  \pagestyle{empty}

  {\huge Simulation of Proof Systems} \hfill Nils Wisiol

  \vspace{3cm}
  
  \begin{definition}
    A function \(h \in \FP\) is called \defNotion{proof system} for a language \(L\) if the range of \(h\) is \(L\). A string \(w\) with \(h(w) = x\) is called an \defNotion{\(h\)-proof} for \(x\).
  \end{definition}

  With this definition, a proof system for \(L\) is basically a polynomial-time bounded function that enumerates \(L\). To give an example, let \(sat\) be defined by
    \[
       sat(x) =
       \begin{cases}
         \varphi & (x = \langle \alpha, \varphi \rangle \text{ and \(\alpha\) is an satisfying assignment for \(\varphi\)}), \\
         \perp & (\text{otherwise}). \\
       \end{cases}
    \]
  Then \(h\) is a proof system for \(\SAT\).

  Notice, in spite of its time bound against the input, the shortest proof of a string \(w \in L\) can be be very long. There may be various proof systems for a language \(L\). In order to make them comparable, we define the notion of \defNotion{simulation} of proof systems.
  
  \begin{definition}
    Let \(h\) and \(h'\) be proof systems for a language \(L\). If there is a polynomial \(p\) and a function \(f\) such that for all \(w \in \Sigma^*\)
      \[
        h(f(w)) = h'(w)
      \]
    and \(|f(w)| \leq p(|w|)\), then \(h\) \defNotion{simulates} \(h'\).
  \end{definition}

  Informally speaking, \(f\) translates \(h\)-proofs into polynomial length bounded \(h'\)-proofs. Notice, \(f\) could be hard or even impossible to calculate.

  \begin{definition}
    A proof system \(h\) for a language \(L\) is called \defNotion{optimal}, if it simulates every proof system for \(L\).
  \end{definition}

  \begin{definition}
    Let \defNotion{\(\OPT\)} be the complexity class of all languages that have an optimal proof system.
  \end{definition}  
  

\end{document}